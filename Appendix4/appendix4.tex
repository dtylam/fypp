%!TEX root = ../thesis.tex
% ******************************* Thesis Appendix B ********************************

\chapter{User Evaluation Interview Transcripts}

Below you will find the sample questions asked for each group, detailed descriptions and relevant transcript
snippets for each problem statements.
Any specifics regarding course, staff and event details have been anonymised.
Here is the participant characterisations table again.

\begin{table}[!h]
	% \caption{Participants in user evaluation interviews}
	\centering
	% \label{table:participants-eval}
	\begin{tabularx}{\textwidth}{>{\bfseries}lX}
		Participant & Characterisation                                                                    \\
		\toprule
		Educator A  & lecturer in higher education for over 20 years, and an experienced higher education
		administrator                                                                                     \\\midrule
		Educator F  & lecturer in higher education for over 10 years                                      \\\midrule
		Student C   & a university course representative for 3 years, which involves collecting and
		communicating student feedback and attending staff-student liaison meetings                       \\\midrule
		Student E   & a university course representative for 2 years and a peer assisted learning leader
		for 1 year                                                                                        \\\bottomrule
	\end{tabularx}
\end{table}

\section{Structured Questions}

These questions are structured as statements, which request for a response first on a Likert-type scale of 1 to 5:
\begin{enumerate}
	\setlength\itemsep{0em}
	\item Strongly disagree
	\item Disagree
	\item Neither agree nor disagree
	\item Agree
	\item Strongly agree
\end{enumerate}
The participant then explains why and how they've come to this opinion, with reference to the demonstrator system.
\newpage
\textbf{Q1. The features of the system communicate assessment expectations very well.}\\
\begin{table}[!ht]
	% \caption{Participants in user evaluation interviews}
	\centering
	% \label{table:participants-eval}
	\begin{tabular}{|c|c|c|c|c|}
		\hline
		Participant & A                       & F                   & C                        & E \\
		\hline
		Scale Given & \cellcolor{Dandelion}3 & \cellcolor{green}5 & \cellcolor{SpringGreen}4 & \cellcolor{SpringGreen}4 \\
		\hline
	\end{tabular}
\end{table}

\textit{Educator A}: "It was clear what the assessment types were... I don't think (the demonstration)
focused on the details of what the assessments required... but in terms of the features of the system,
yes it does have the potential to communicate (the expectations) well. If (the assessment brief) is
poorly written then it does not, if it is well written than it does, that is not down to the function of the system."

\textit{Educator F}: "I think this is more than we currently provide to our students... it would be nice if
(the list of knowledge required) has direct links to particular lectures or materials."

\textit{Student E}: "When it shows the objectives... before they upload the submission... 
so they can't say they've forgotten about it. But it is a bit too much in your face, it would stress me out."\\

\textbf{Q2. The features of the system improve transparency in assessment procedures.}\\
\begin{table}[!ht]
	% \caption{Participants in user evaluation interviews}
	\centering
	% \label{table:participants-eval}
	\begin{tabularx}{0.325\textwidth}{|c|c|c|c|c|}
		\hline
		Participant & A                      & F                  & C                  & E \\
		\hline
		Scale Given & \cellcolor{Dandelion}3 & \cellcolor{green}5 & \cellcolor{green}5 & \cellcolor{green}5 \\
		\hline
	\end{tabularx}
\end{table}

\textit{Educator A}: "Current systems can do it very badly or very well. It makes it more transparent than
the systems that don't provide that level of transparency already now. Some of them already do. For example,
you can build the criteria-based marking scheme in your system into other systems. If you force people to do that
in a consistent way, yes it does."

\textit{Educator F}: "It absolutely does...
especially when a student can see what the assessor has picked (for each criteria)."

\textit{Student E}: "I think the marking criteria, and seeing what we got from the marker is really useful 
because we don't have that currently. I think that is really good."\\

\textbf{Q3. The features of the system make curriculum personalisation convenient.}\\
\begin{table}[!ht]
	% \caption{Participants in user evaluation interviews}
	\centering
	% \label{table:participants-eval}
	\begin{tabularx}{0.325\textwidth}{|c|c|c|c|c|}
		\hline
		Participant & A                  & F                  & C                  & E \\
		\hline
		Scale Given & \cellcolor{green}5 & \cellcolor{green}5 & \cellcolor{green}5 & \cellcolor{SpringGreen}4 \\
		\hline
	\end{tabularx}
\end{table}

\textit{Educator A}: "Yes it does if what you are doing is collecting blocks. It was presented in a way that was
easy to see what it was."

\textit{Student E}: "(I like that) only your chosen modules are shown (in the Ongoing Modules page). At the moment, our system is very crowded."\\

\textbf{Q4. The features of the system provide good (administrative/ pastoral) support for curriculum personalisation.}\\
\begin{table}[!ht]
	% \caption{Participants in user evaluation interviews}
	\centering
	% \label{table:participants-eval}
	\begin{tabularx}{0.325\textwidth}{|c|c|c|c|c|}
		\hline
		Participant & A                  & F                  & C                  & E \\
		\hline
		Scale Given & \cellcolor{green}5 & \cellcolor{green}5 & \cellcolor{SpringGreen}4 & \cellcolor{Dandelion}3 \\
		\hline
	\end{tabularx}
\end{table}

\textit{Educator A}: "I think the interesting thing is the ability to chat with advisors about them.
There is also the potential for the chats to be stored.
I think having that (chat) record that can be revisited is a good thing as long as (it is kept private)."

\textit{Educator F}: "I like the fact that you can pick your modules while asking your tutor or support staff for more advice.
It could be administrative issues as well to the support staff. When I see (my students) choose their modules they sometimes
choose it blindly because they don't really have an idea what they want, so this could provide a lot more information and support.
It is actually good for the organisation as well... having a more accurate headcount as the students are more confident about their choices."

\textit{Student C}: "It's not a lot of admin work and it's a set of straightforward work, but you will need a few more people
ready to support."

\textit{Student E}: "I think it should be your decision what modules you want to do. The titles and programme outcomes... 
the system should just decide automatically based on the modules you chose. I don't want to give my supervisor the power 
to approve my decisions when I am paying (tuition)."\\

\textbf{Q5. The system can reduce tension and disagreements between educators and students.}\\
\begin{table}[!ht]
	% \caption{Participants in user evaluation interviews}
	\centering
	% \label{table:participants-eval}
	\begin{tabularx}{0.325\textwidth}{|c|c|c|c|c|}
		\hline
		Participant & A                        & F                      & C                      & E \\
		\hline
		Scale Given & \cellcolor{SpringGreen}4 & \cellcolor{Dandelion}3 & \cellcolor{Dandelion}3 & \cellcolor{SpringGreen}4 \\
		\hline
	\end{tabularx}
\end{table}

\textit{Educator A}: "I think these things are best measured in practice. I think it would because of the focus on
criteria-based statements... it doesn't stop people from arguing with the criteria, (students) are not really interested
in criteria they are interested in grades. So it wouldn't get rid of disagreements and tension, but it would reduce it
and focus the discussion on the criteria."

\textit{Educator F}: "(The features) increase the provenance and transparency in how we grade and mark,
that's a very good thing and that's what blockchain is all about, but I am not sure if this reduces the tension.
I think it will most likely help people resolve disagreements, not to reduce them to begin with."

\textit{Student C}: "If there is a disagreement, the student cannot report back in the system. The student should be
able to start a conversation and argue their case... that should be on the blockchain."

\textit{Student E}: "Somewhat, having the marking criteria and objectives on the page of submission means that they are the most  
recent version when you submit. The feedback on marking scheme also makes more sense when you read it. 
But it (could) stress students out more."\\

\textbf{Q6. The features of the system make educational history more transparent and trustworthy.}\\
\begin{table}[!ht]
	% \caption{Participants in user evaluation interviews}
	\centering
	% \label{table:participants-eval}
	\begin{tabularx}{0.325\textwidth}{|c|c|c|c|c|}
		\hline
		Participant & A                  & F                  & C                      & E \\
		\hline
		Scale Given & \cellcolor{green}5 & \cellcolor{green}5 & \cellcolor{Dandelion}3 & \cellcolor{green}5 \\
		\hline
	\end{tabularx}
\end{table}

\textit{Educator A}: "Yes it does because you've got immutable records."

\textit{Student C}: "If your conversations go wrong, you lose trust. Sometimes when people speak face to face
it is a different interaction than what they do on a computer."

\textit{Student E}: "It is very transparent. It is black and white. If you are a recruiter, you'd know it is official 
and not made up."\\

\textbf{Q7. The access control features of the system preserve student privacy.}\\
\begin{table}[!ht]
	% \caption{Participants in user evaluation interviews}
	\centering
	% \label{table:participants-eval}
	\begin{tabularx}{0.325\textwidth}{|c|c|c|c|c|}
		\hline
		Participant & A                        & F                  & C                  & E \\
		\hline
		Scale Given & \cellcolor{SpringGreen}4 & \cellcolor{green}5 & \cellcolor{green}5 & \cellcolor{green}5 \\
		\hline
	\end{tabularx}
\end{table}

\textit{Educator A}: "That depends on the security of the blockchain. Also, it is not private between you and
the grader, or (instituional administrators)... privacy is not a right in all cases. (For the public readers,)
I think that's a really great idea, some universities are starting to do systems were you could make your transcript
available to people, (this system) puts you in control of it."

\textit{Educator F}: "Yes because you are allowed to set different levels of access based on someone's role and
that is essential."

\textit{Student C}: "I like that you've got a preview of what others can see and you can tap on and off."\\

\textbf{Q8. The features of the system increase trust in online education providers and credentials.}\\
\begin{table}[!ht]
	% \caption{Participants in user evaluation interviews}
	\centering
	% \label{table:participants-eval}
	\begin{tabularx}{0.325\textwidth}{|c|c|c|c|c|}
		\hline
		Participant & A                        & F                  & C                        & E \\
		\hline
		Scale Given & \cellcolor{SpringGreen}4 & \cellcolor{green}5 & \cellcolor{SpringGreen}4 & \cellcolor{SpringGreen}4 \\
		\hline
	\end{tabularx}
\end{table}

\textit{Educator A}: "If you are aggregating content from anywhere,
the quality of the content and assessments is the issue. The question is who is issuing the award,
because often credibility and trust is about the award issuer.
An institution might be more interested in protecting their reputation and use an instance of the blockchain
instead of participating in the global marketplace blockchain. A MOOC platform like Coursera will have to issue
their own awards. You cannot rely on the blockchain to give trust, unless you've got a crowd-sourced model,
where students and academics can review and rate courses."
% For existing (MOOC platforms) like coursera putting up their existing content, 
% the quality of the content is the issue... for an institution offering courses through this platform, 
% the credibility of the institution is the issue. 

\textit{Educator F}: "I think if you ask me a year ago I wouldn't (be convinced), but now with bitcoin and
blockchain everywhere I think people are beginning to understand and it is becoming more mainstream in a way.
People have developed trust because they can understand the technology better."

\textit{Student C}: "There is clarity... and the simplicity of it. You can see everything. Some systems get rid
of your records after a period of time."\\

\section{Semi-Structured Questions}

\textbf{Q9. What are your thoughts on how the system conducts assessments and curriculum personalisation?}\\

\textit{Educator A}: "The system poses a lot of constraints on how assessment can be conducted.
The assessor should be able to override grades the rubric has calculated with an explicit justification...
that is transparent to the student."

\textit{Educator F}: "I think it will not replace existing systems, but as universities start to offer more
courses online to reduce cost and provide flexibility to students. I can see this system being extremely useful
for that situation."

\textit{Educator F}: "For traditional campus students, it could add value because it integrates functions of different systems,
such as blackboard (a course content delivery platform) and Wiseflow (a digital assessment platform)."

\textit{Educator F}: "There could be resistance from a lot of module leaders (on having detailed marking schemes)."

\textit{Student E}: "I do want to see my supervisors face to face. If everything happens on the system, 
where is the trust in that? I want to know if they are behind that screen... a lot of students may not be comfortable 
with video chats and recordings."\\

\textbf{Q10. Is the system useful? Would you consider enrolling into this marketplace platform in the future?}\\

\textit{Educator A}: "As a provider of content, yes. I think the market globally is going towards
individual providers of content. I think the UK is not the place of such a market but the US is.
Even for physical universities, that would be a good thing because they can recruit more students globally, and
take the pressure off their estate."

\textit{Educator A}: "I know of a case where a student from 20 years ago has came back to ask for his records
and a transcript, but the department no longer has them (after moving systems and records several times).
I think the student is trying to pull a fast one where he has never graduated. Recording assessments has massive value,
and gives much greater security."

\textit{Educator F}: "Potentially you can pick credits from Stanford and Harvard.
It makes sense from the student's point of view, and also to people who may not have access to institutions.
I would use it as a teacher as well."

\textit{Student C}: "Yes, just because it is easier and more convenient."

\textit{Student E}: "Yes I absolutely would. It makes you feel like every student has their own thing and 
only you could see this. And I like the user experience as well."
