%!TEX root = ../thesis.tex
% ******************************* Thesis Appendix B ********************************

\chapter{User Evaluation Interview Transcripts}

Below you will find the sample questions asked for each group, detailed descriptions and relevant transcript 
snippets for each problem statements. 
Any specifics regarding course, staff and event details have been anonymised.
Here is the participant characterisations table again.

\begin{table}[!h] 
    % \caption{Participants in user evaluation interviews}
    \centering
    % \label{table:participants-eval}
    \begin{tabularx}{\textwidth}{>{\bfseries}lX}
        Participant & Characterisation\\
        \toprule
        Educator A & lecturer in higher education for over 20 years, and an experienced higher education 
        administrator\\\midrule
        Educator F & lecturer in higher education for over 10 years\\\midrule
        Student C & a university course representative for 3 years, which involves collecting and 
        communicating student feedback and attending staff-student liaison meetings \\\midrule
        Student E & a university course representative for 2 years and a peer assisted learning leader 
        for 1 year\\\bottomrule
    \end{tabularx}
\end{table}

\section{Structured Questions}

These questions are structured as statements, which request for a response first on a Likert-type scale of 1 to 5:
\begin{enumerate}
	\setlength\itemsep{0em}    
    \item Strongly disagree
    \item Disagree
    \item Neither agree nor disagree 
    \item Agree
    \item Strongly agree
\end{enumerate}
The participant then explains why and how they've come to this opinion, with reference to the demonstrator system.
\newpage
\textbf{Q1. The features of the system communicate assessment expectations very well.}\\
\begin{table}[!ht] 
    % \caption{Participants in user evaluation interviews}
    \centering
    % \label{table:participants-eval}
    \begin{tabularx}{0.325\textwidth}{|c|c|c|c|c|}
        \hline
        Participant & A & F & C & E \\
        \hline
        Scale Given & 3 & ? & ? & ? \\
        \hline
    \end{tabularx}
\end{table}

\textit{Educator A}: "It was clear what the assessment types were... I don't think (the demonstration) 
focused on the details of what the assessments required... but in terms of the features of the system, 
yes it does have the potential to communicate (the expectations) well. If (the assessment brief) is 
poorly written then it does not, if it is well written than it does, that is not down to the function of the system."\\

\textbf{Q2. The features of the system improve transparency in assessment procedures.}\\
\begin{table}[!ht] 
    % \caption{Participants in user evaluation interviews}
    \centering
    % \label{table:participants-eval}
    \begin{tabularx}{0.325\textwidth}{|c|c|c|c|c|}
        \hline
        Participant & A & F & C & E \\
        \hline
        Scale Given & 3 & ? & ? & ? \\
        \hline
    \end{tabularx}
\end{table}

\textit{Educator A}: "Current systems can do it very badly or very well. It makes it more transparent than 
the systems that don't provide that level of transparency already now. Some of them already do. For example, 
you can build the criteria-based marking scheme in your system into other systems. If you force people to do that 
in a consistent way, yes it does."\\

\textbf{Q3. The features of the system make curriculum personalisation convenient.}\\
\begin{table}[!ht] 
    % \caption{Participants in user evaluation interviews}
    \centering
    % \label{table:participants-eval}
    \begin{tabularx}{0.325\textwidth}{|c|c|c|c|c|}
        \hline
        Participant & A & F & C & E \\
        \hline
        Scale Given & 5 & ? & ? & ? \\
        \hline
    \end{tabularx}
\end{table}

\textit{Educator A}: "Yes it does if what you are doing is collecting blocks. It was presented in a way that was 
easy to see what it was."\\

\textbf{Q4. The features of the system provide good (administrative/ pastoral) support for curriculum personalisation.}\\
\begin{table}[!ht] 
    % \caption{Participants in user evaluation interviews}
    \centering
    % \label{table:participants-eval}
    \begin{tabularx}{0.325\textwidth}{|c|c|c|c|c|}
        \hline
        Participant & A & F & C & E \\
        \hline
        Scale Given & 5 & ? & ? & ? \\
        \hline
    \end{tabularx}
\end{table}

\textit{Educator A}: "I think the interesting thing is the ability to chat with advisors about them. 
There is also the potential for the chats to be stored. 
I think having that (chat) record that can be revisited is a good thing as long as (it is kept private)."\\

\textbf{Q5. The system can reduce tension and disagreements between educators and students.}\\
\begin{table}[!ht] 
    % \caption{Participants in user evaluation interviews}
    \centering
    % \label{table:participants-eval}
    \begin{tabularx}{0.325\textwidth}{|c|c|c|c|c|}
        \hline
        Participant & A & F & C & E \\
        \hline
        Scale Given & 4 & ? & ? & ? \\
        \hline
    \end{tabularx}
\end{table}

\textit{Educator A}: "I think these things are best measured in practice. I think it would because of the focus on 
criteria-based statements... it doesn't stop people from arguing with the criteria, (students) are not really interested 
in criteria they are interested in grades. So it wouldn't get rid of disagreements and tension, but it would reduce it 
and focus the discussion on the criteria."

\textbf{Q6. The features of the system make educational history more transparent and trustworthy.}\\
\begin{table}[!ht] 
    % \caption{Participants in user evaluation interviews}
    \centering
    % \label{table:participants-eval}
    \begin{tabularx}{0.325\textwidth}{|c|c|c|c|c|}
        \hline
        Participant & A & F & C & E \\
        \hline
        Scale Given & 5 & ? & ? & ? \\
        \hline
    \end{tabularx}
\end{table}

\textit{Educator A}: "Yes it does because you've got immutable records."\\

\textbf{Q7. The features of the system preserve student privacy.}\\
\begin{table}[!ht] 
    % \caption{Participants in user evaluation interviews}
    \centering
    % \label{table:participants-eval}
    \begin{tabularx}{0.325\textwidth}{|c|c|c|c|c|}
        \hline
        Participant & A & F & C & E \\
        \hline
        Scale Given & 4 & ? & ? & ? \\
        \hline
    \end{tabularx}
\end{table}

\textit{Educator A}: "That depends on the security of the blockchain. Also, it is not private between you and 
the grader, or (instituional administrators)... privacy is not a right in all cases. (For the public readers,) 
I think that's a really great idea, some universities are starting to do systems were you could make your transcript 
available to people, (this system) puts you in control of it."\\

\textbf{Q8. The features of the system increase trust in online education providers and credentials.}\\
\begin{table}[!ht] 
    % \caption{Participants in user evaluation interviews}
    \centering
    % \label{table:participants-eval}
    \begin{tabularx}{0.325\textwidth}{|c|c|c|c|c|}
        \hline
        Participant & A & F & C & E \\
        \hline
        Scale Given & 4 & ? & ? & ? \\
        \hline
    \end{tabularx}
\end{table}

\textit{Educator A}: "If you are aggregating content from anywhere, 
the quality of the content and assessments is the issue. The question is who is issuing the award, 
because often credibility and trust is about the award issuer. 
An institution might be more interested in protecting their reputation and use an instance of the blockchain 
instead of participating in the global marketplace blockchain. A MOOC platform like Coursera will have to issue 
their own awards. You cannot rely on the blockchain to give trust, unless you've got a crowd-sourced model, 
where students and academics can review and rate courses."\\
% For existing (MOOC platforms) like coursera putting up their existing content, 
% the quality of the content is the issue... for an institution offering courses through this platform, 
% the credibility of the institution is the issue. 

\section{Semi-Structured Questions}

\textbf{Q9. What are your thoughts on how the system conducts assessments?}\\

\textit{Educator A}: "The system poses a lot of constraints on how assessment can be conducted. 
The assessor should be able to override grades the rubric has calculated with an explicit justification... 
that is transparent to the student."

\textbf{Q10. What are your thoughts on how the system enables curriculum personalisation easier?}\\

\textbf{Q11. Is the system useful? Would you consider enrolling into this marketplace platform in the future?}\\

\textit{Educator A}: "As a provider of content, yes. I think the market globally is going towards 
individual providers of content. I think the UK is not the place of such a market but the US is. 
Even for physical universities, that would be a good thing because they can recruit more students globally, and 
take the pressure off their estate."

\textit{Educator A}: "I know of a case where a student from 20 years ago has came back to ask for his records 
and a transcript, but the department no longer has them (after moving systems and records several times). 
I think the student is trying to pull a fast one where he has never graduated. Recording assessments has massive value, 
and gives much greater security."