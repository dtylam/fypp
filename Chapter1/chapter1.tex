%!TEX root = ../thesis.tex
%*******************************************************************************
%*********************************** First Chapter *****************************
%*******************************************************************************

\chapter{Introduction}  %Title of the First Chapter

% Provide a brief introduction to your project, providing some background which allows you to clearly present
%  the problem that you are seeking to address in your dissertation.  This section should prepare the reader 
%  for the Aims and Objectives which come next. 

% You may draw on some of your background study as evidence, but you should leave the full background discussion to chapter 2.

The global e-Learning industry already generates US\$60 billion per year, and by 2019, over half of all courses will be taken 
online \citep[p.17]{panto2013challenge}. This rising trend presents an opportunity to improve higher education.

Some current problems in higher education are related to transparency (more in Chapter 2.2.1). Tension exists between 
the educational provider and the learners over assessments. "There is abundant evidence that assessors are not particularly 
good at making exams valid, reliable, or transparent to students." \citep[p.62]{brown1999assessment}.

% Accountability and transparency is important especially in higher education, which subscribes to an audit-based quality control lifecycle (Hoecht, 2006).
% Employers have a vested interest in what is assessed and the fairness of assessments in education, because it affects the recruitment of employees (Brown, 1999, p.58).

There is also a lack of curriculum personalisation for higher education learners in the UK [TODO: cite interviews, other refs] 
\citet{condie2007impact} pointed out that the personalisation of the education curriculum for learners helps "overcome 
barriers, raising self-esteem and achievement". 
% Current web, mobile and computer technologies today can provide more personalisation of education curricula, but lacks
% [TODO: common marketplace? promise of delivery? transparency for employers?]

Being able to deliver education curricula and conduct assessments in a transparent, conflict-free way would be central to 
a future e-Learning marketplace that is open, trusted and autonomous.
This is where immediate value could be provided by blockchain systems and smart contracts.

A blockchain is a type of database that is spread across multiple sites, such as different institutions, companies 
or participants. A "block" of records is "chained" to the next with a cryptographic signature, creating permanent records 
through a consensus corroborated by all the operators \citep[p.17]{walport2016distributed}.
The verifiability of all actions on a blockchain gives systems an inherently high degree of integrity and transparency. 
Some types of blockchain can also have permissions that enables granular transparency and privacy \citep[p.22]{walport2016distributed}, 
which fits well in the education paradigm where personal data can be highly sensitive.

Smart contracts are "contracts" that are "defined by the code and executed (or enforced) by the code" \citep[p.16]{swan2015blockchain}.
They are logic embedded in a blockchain that defines the rules and penalties around an agreement and could automatically enforce 
those obligations \citep{gulhane2017ibm}, and can be used to exchange or transfer digital assets when certain conditions are met. 
% They should be autonomous, self-sufficient, and decentralised.

The potential of blockchain-enabled systems in education has been noted by the community, with \citet[p.62]{swan2015blockchain} 
proposing that “learning smart contracts could automatically confirm the completion of learning modules through standardized 
online tests”. Appropriate configurations in permissions and visibility can also provide improved security and privacy to e-Learning.

%********************************** %First Section  **************************************
\section{Aims and Objectives} %Section - 1.1 

The aim of the project is to design an e-Learning platform that fulfills educational assessments and 
rewards with smart contracts on a blockchain, providing improvements in assessments and curriculum 
personalisation for learners and teachers.

To satisfy this aim, the following objectives are planned:

\begin{enumerate}
    \item Identify issues in education and e-Learning that can be improved by a blockchain-based system.
    \item Develop data models and smart contracts in the proposed blockchain for e-Learning.
    \item Develop permission models for the e-Learning blockchain that balances appropriate access and privacy protection.
    \item Build a demonstrator system that includes client-side applications for learners and teachers.
    \item Evaluate whether the blockchain-based demonstrator tackles the issues in education and e-Learning identified in objective 1
\end{enumerate}

    % 2. Identify an approach which, when executed, will give rise to results from which rigorous conclusions can be drawn.
    % 3. Design and implement some software, or undertake a simulation, or business modeling exercise, or conduct some other kind of appropriate activity which will give rise to the results desired.
    % 5. Evaluate the results using an appropriate framework, or set of success criteria which are clearly related to the problem and stated aim.


%********************************** %Second Section  *************************************
\section{Project Approach} %Section - 1.2

\begin{enumerate}
    \item Review literature on current issues in e-Learning and education, and existing work in blockchain in education.
    \item Further gather requirements for a blockchain solution for e-Learning using interviews with stakeholders.
    \item Design data models, smart contracts and permission rules for assets and participants in the proposed blockchain solution.
    \item Analyse popular blockchain development platforms that can be used to produce the desired solution.
    \item Build the distributed ledger network and client side applications for learners and teachers.
    \item Evaluate the design of the deliverables using interviews with stakeholders and relevant subject matter experts.
\end{enumerate}

%********************************** % Third Section  *************************************
\section{Dissertation Outline}  %Section - 1.3 

Chapter 2 discusses the background for my project, and identifies some key techniques that can be adopted during the development 
of the proposed solution.  

Chapter 3 explains how the project will be undertaken . . . etc, etc.  

Chapter 4 requirements incl. primary data from interviews

Chapter 5 design

Chapter 6 implementation

Chapter 7 evaluation

Chapter 8 conclusion, future work

% This approach is acceptable, however it can make quite bland reading.  You might like to consider drawing a flow-chart of your 
% project, showing how information such as background data, questionnaire data, results of studies, running computer programs, or 
% undertaking user studies act as input to, or output from your chapters. You can also indicate how each chapter relates to your objectives.  This kind of diagram can help to add clarity for your reader, and can help you to get your head round the structure of your project.

% Lorem Ipsum is simply dummy text of the printing and typesetting industry (see 
% Section~\ref{section1.3}). Lorem Ipsum~\citep{Aup91} has been the industry's 
% standard dummy text ever since the 1500s, when an unknown printer took a galley 
% of type and scrambled it to make a type specimen book. It has survived not only 
% five centuries, but also the leap into electronic typesetting, remaining 
% essentially unchanged. It was popularised in the 1960s with the release of 
% Letraset sheets containing Lorem Ipsum passages, and more recently with desktop 
% publishing software like Aldus PageMaker including versions of Lorem 
% Ipsum~\citep{AAB95,Con90,LM65}.

% The most famous equation in the world: $E^2 = (m_0c^2)^2 + (pc)^2$, which is 
% known as the \textbf{energy-mass-momentum} relation as an in-line equation.

% A {\em \LaTeX{} class file}\index{\LaTeX{} class file@LaTeX class file} is a file, which holds style information for a particular \LaTeX{}.


% \begin{align}
% CIF: \hspace*{5mm}F_0^j(a) = \frac{1}{2\pi \iota} \oint_{\gamma} \frac{F_0^j(z)}{z - a} dz
% \end{align}

% \nomenclature[z-cif]{$CIF$}{Cauchy's Integral Formula}                                % first letter Z is for Acronyms 
% \nomenclature[a-F]{$F$}{complex function}                                                   % first letter A is for Roman symbols
% \nomenclature[g-p]{$\pi$}{ $\simeq 3.14\ldots$}                                             % first letter G is for Greek Symbols
% \nomenclature[g-i]{$\iota$}{unit imaginary number $\sqrt{-1}$}                      % first letter G is for Greek Symbols
% \nomenclature[g-g]{$\gamma$}{a simply closed curve on a complex plane}  % first letter G is for Greek Symbols
% \nomenclature[x-i]{$\oint_\gamma$}{integration around a curve $\gamma$} % first letter X is for Other Symbols
% \nomenclature[r-j]{$j$}{superscript index}                                                       % first letter R is for superscripts
% \nomenclature[s-0]{$0$}{subscript index}                                                        % first letter S is for subscripts
