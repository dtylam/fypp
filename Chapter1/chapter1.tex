%!TEX root = ../thesis.tex
%*******************************************************************************
%*********************************** First Chapter *****************************
%*******************************************************************************

\chapter{Introduction}  %Title of the First Chapter

% Provide a brief introduction to your project, providing some background which allows you to clearly present
%  the problem that you are seeking to address in your dissertation.  This section should prepare the reader 
%  for the Aims and Objectives which come next. 

% You may draw on some of your background study as evidence, but you should leave the full background discussion to chapter 2.

The global e-Learning industry already generates US\$60 billion per year, and by 2019, over half of all courses will be taken 
online \citep[p.17]{panto2013challenge}. This rising trend presents an opportunity to improve higher education.

Some current problems in higher education are related to transparency. [TODO: what is transparency?] Tension exists between 
the educational provider and the learners over assessments. "There is abundant evidence that assessors are not particularly 
good at making exams valid, reliable, or transparent to students." \citep[p.62]{brown1999assessment}.

% Accountability and transparency is important especially in higher education, which subscribes to an audit-based quality control lifecycle (Hoecht, 2006).
% Employers have a vested interest in what is assessed and the fairness of assessments in education, because it affects the recruitment of employees (Brown, 1999, p.58).

There is also a lack of curriculum personalisation for higher education learners in the UK [TODO: due to..., ref Rob] 
\citet{condie2007impact} pointed out that the personalisation of the education curriculum for learners helps "overcome 
barriers, raising self-esteem and achievement". 
Current web, mobile and computer technologies today can provide more personalisation of education curricula, but lacks
[TODO: common marketplace? promise of delivery? transparency for employers?]

Being able to deliver education curricula and conduct assessments in a transparent, conflict-free way would be central to 
a future e-Learning marketplace that is open, trusted and autonomous.
This is where immediate value could be provided by distributed ledger systems and smart contracts.

A distributed ledger is a type of database that is spread across multiple sites, such as different institutions, companies 
or participants. Validators or operators of this ledger are trusted not to collude and defraud actors in a transaction. 
The technology enabling this distributed ledger is popularly known as a blockchain, 
where a block of records is chained to the next with a cryptographic signature, creating immutable records through a consensus
corroborate by all the operators. \citep[p.17]{walport2016distributed}
The security, immutability and verifiability of all actions on a blockchain provides the system with maximal transparency.

Smart contracts are "contracts" that are "defined by the code and executed (or enforced) by the code" \citep[p.16]{swan2015blockchain}.
They are logic embedded in a blockchain that defines the rules and penalties around an agreement and automatically enforce 
those obligations \citep{gulhane2017ibm}, and can be used to exchange or transfer digital assets when certain conditions are met. 
% They should be autonomous, self-sufficient, and decentralised.

The potential of blockchain enabled systems in education has been noted by the community, with \citet[p.62]{swan2015blockchain} 
proposing that “learning smart contracts could automatically confirm the completion of learning modules through standardized 
online tests”. Appropriate configurations in permissions and visibility can also provide improved security and privacy to e-Learning.

%********************************** %First Section  **************************************
\section{Aims and Objectives} %Section - 1.1 

Designing a system that fulfill educational assessments and rewards with publicly visible smart contracts on a blockchain.

The ideal system will enable the exchange of modular smart contracts provided by education providers, executed by the peer network. This will facilitate the negotiation and fulfillment of personalised learning plans, guarantee the openness and fairness of assessment and rewards, and increase trust in educational credentials.
The project will:
-	Discuss the role of smart contracts in e-Learning
-	Design smart contracts for the proposed scenario
-	Build a demonstrator that includes:
a small network of peers hosted on containers
one example course containing two or more modular smart contracts
a client-side application that allows e-learners to invoke learning actions
a public facing application that allows queries to e-Learning smart contracts

Here you should clearly define the overarching aim for your project.  Usually, for a final year project, you will have a single aim.

You should then list, the necessary and complete set of objectives that you will need to achieve in order to satisfy the aim:

    1. Undertake a relevant background study to identify existing work in the area, and to identify appropriate techniques which can be adopted to produce a solution in this project.
    2. Identify an approach which, when executed, will give rise to results from which rigorous conclusions can be drawn.
    3. Design and implement some software, or undertake a simulation, or business modeling exercise, or conduct some other kind of appropriate activity which will give rise to the results desired.
    4. Tailor the generic objectives to make them relevant for your specific project.  Generic aims and objectives will lead to low-grading, generic project.
    5. Evaluate the results using an appropriate framework, or set of success criteria which are clearly related to the problem and stated aim.


%********************************** %Second Section  *************************************
\section{Project Approach} %Section - 1.2
Describe how the project will be undertaken.  Remember that the way in which you conduct your project will dictate the nature of the results that you produce, and the corresponding conclusions you can draw from them.  This is why it is important that your reader understands how you are going about your project from an early stage, so they can understand how to interpret your results.
• Review literature on e-Learning and assessments
• Design the parameters and functions of the smart contract on the IBM Hyperledger Composer
• Implement the system on a local Docker peer cluster
• Implement the learner and public facing applications

%********************************** % Third Section  *************************************
\section{Dissertation Outline}  %Section - 1.3 
Traditionally, dissertations tend to contain a description of each chapter:

Chapter 2, discusses the background for my project, and identifies some key techniques that can be adopted during the development of the proposed solution.  Chapter 3 explains how the project will be undertaken . . . etc, etc.  

This approach is acceptable, however it can make quite bland reading.  You might like to consider drawing a flow-chart of your project, showing how information such as background data, questionnaire data, results of studies, running computer programs, or undertaking user studies act as input to, or output from your chapters. You can also indicate how each chapter relates to your objectives.  This kind of diagram can help to add clarity for your reader, and can help you to get your head round the structure of your project.

% Lorem Ipsum is simply dummy text of the printing and typesetting industry (see 
% Section~\ref{section1.3}). Lorem Ipsum~\citep{Aup91} has been the industry's 
% standard dummy text ever since the 1500s, when an unknown printer took a galley 
% of type and scrambled it to make a type specimen book. It has survived not only 
% five centuries, but also the leap into electronic typesetting, remaining 
% essentially unchanged. It was popularised in the 1960s with the release of 
% Letraset sheets containing Lorem Ipsum passages, and more recently with desktop 
% publishing software like Aldus PageMaker including versions of Lorem 
% Ipsum~\citep{AAB95,Con90,LM65}.

% The most famous equation in the world: $E^2 = (m_0c^2)^2 + (pc)^2$, which is 
% known as the \textbf{energy-mass-momentum} relation as an in-line equation.

% A {\em \LaTeX{} class file}\index{\LaTeX{} class file@LaTeX class file} is a file, which holds style information for a particular \LaTeX{}.


% \begin{align}
% CIF: \hspace*{5mm}F_0^j(a) = \frac{1}{2\pi \iota} \oint_{\gamma} \frac{F_0^j(z)}{z - a} dz
% \end{align}

% \nomenclature[z-cif]{$CIF$}{Cauchy's Integral Formula}                                % first letter Z is for Acronyms 
% \nomenclature[a-F]{$F$}{complex function}                                                   % first letter A is for Roman symbols
% \nomenclature[g-p]{$\pi$}{ $\simeq 3.14\ldots$}                                             % first letter G is for Greek Symbols
% \nomenclature[g-i]{$\iota$}{unit imaginary number $\sqrt{-1}$}                      % first letter G is for Greek Symbols
% \nomenclature[g-g]{$\gamma$}{a simply closed curve on a complex plane}  % first letter G is for Greek Symbols
% \nomenclature[x-i]{$\oint_\gamma$}{integration around a curve $\gamma$} % first letter X is for Other Symbols
% \nomenclature[r-j]{$j$}{superscript index}                                                       % first letter R is for superscripts
% \nomenclature[s-0]{$0$}{subscript index}                                                        % first letter S is for subscripts
