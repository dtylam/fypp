%!TEX root = ../thesis.tex
%*******************************************************************************
%****************************** Seventh Chapter **********************************
%*******************************************************************************
\chapter{Conclusion}

\section{Work Completed}

The main deliverables of this project have been the demonstrator system, the 
design work on blockchain data models, Smart Contracts and access control rules, 
and the primary data collected in requirements gathering and evaluation.

Going back to the aim of the project (as in Chapter 1.1), an e-Learning platform 
that fulfils educational assessments has been successfully designed and built. 
Both teachers and students have responded positively to the improvements made 
in assessments and curriculum personalisation. They have also agreed that it could 
improve privacy, security, trust, and help resolve conflicts in education.

The aim of the project has largely been satisfied, 
and it has been a good proof of the benefits that blockchain systems 
and Smart Contracts can bring to education.

\section{Limitations}

The potential improvements in communication provided by the system were limited 
by human actors. Quality of information and interactions are still dependant on  
individual teachers and students.
For example, \citet{bryan2006innovative} noted that the explicit descriptions of 
learning outcomes, marking criteria and grade descriptors are not enough in higher education,
that teachers should engaging students in pre-assessment activities that create a 
social construct of what those statements mean.

The highly regulated and structured context of high education may have been limiting 
to this project. For example, institutions could be reluctant to liberalise their 
degree structures, or advocate automatic assessments, 
in order to protect their academic standards and reputation. 
A broader, general e-Learning platform for MOOCs or skill-sharing 
may have been a better fit for blockchain technologies.

Lastly, public awareness of the actual functionalities and benefits of blockchains 
and Smart Contracts seemed to stop at the belief that they are more secure. Properties 
such as peer execution and consensus were not well understood. The user interfaces of 
the demonstrator applications should have done more to illustrate these concepts 
through animations and descriptions.

\section{Future Work}

There are many directions in which future work can go. Some of the ideas are:

\begin{itemize}
    \item \textbf{Create more higher education features}: 
    several features were unfinished and many more were suggested in user evaluation. 
    For example: certificates and transcripts; more assessment paradigms such as peer assessments, 
    self assessments, double marking; features that support arbitration, 
    appeals and extenuating circumstances for assessments.
    \item \textbf{Open-source development}:
    further develop the demonstrator system to create a versatile e-Learning software 
    platform that any interested developers can clone and fork for their own purposes. 
    This requires many improvements and may require changes to the current dependency on 
    Hyperledger Composer and Vue.js to overcome technical limitations.
    \item \textbf{Create a proprietary MOOC platform}:
    changing some of the designs to fit the Massive Open Online Course market 
    and productionising it, setting up a company to attract funding and content creators. 
    This again will require many improvements to the system built so far.
    \item \textbf{Create a UI component and styling framework for blockchain applications}:
    the effectiveness of this project was partially dampened by the lack of 
    an effective visual UI framework for blockchain and Smart Contract concepts/ features.
    \item \textbf{Investigate the role of cryptocurrency and micropayments in education}:
    Smart Contracts could also create a cryptocurrency and distribute it to participants, 
    there could be value in creating a cryptocurrency in education.
\end{itemize}

Public interest in blockchains has continued to rise, and the use of blockchain in education continues 
to gain traction. Most recently in March 2018, a group of academics at Oxford University launched 
a for-profit online university using a blockchain back-end as a selling point \citep{pells2018blockchainuni}. 
While not mainstream yet, it is certain that the use of blockchains and Smart Contracts will 
have a foothold in the education sector in the near future.

% - tests embedded in smart contracts instead of rest calls, which may not always be available

% - consensus model for double marking, etc

% -- reputation model and community policing build trust

% -- who issues the award? different modes and uses. 

% -- 



% and here I write more \dots

% https://www.timeshighereducation.com/news/oxford-academics-launch-worlds-first-blockchain-university