%!TEX root = ../thesis.tex
%*******************************************************************************
%****************************** Seventh Chapter **********************************
%*******************************************************************************
\chapter{Evaluation}

Coming to the end of the project, evaluation is needed to measure and draw conclusions
on its effectiveness. The project will be evaluated in two ways:
software testing and user evaluation.

\section{Software Testing}

One way to determine product quality is to evaluate the extent to which each stated requirement 
has been fulfilled \citep{bach1999risk}. Since the requirements for this project addresses directly 
several of the problems found in literature and interviews with stakeholders, a high quality product 
that satisfies the requirements would make the project more of a success.

This can be done with black box testing, with test cases that are traceable to one or more of 
the requirements as stated in Chapter 4.2.

The full list of test cases run are listed in Appendix E. The test results are overall positive. 
Out of [TODO], test cases were not passing, and here are the remedial actions.

A limitation of these tests are that they are performed by the same person as the developer and could be biased.

\section{User Evaluation}

The aims of the project aims to deliver benefits to students and teachers. 
It is therefore important to evaluate what these real world stakeholders think about the project deliverables.

\subsection{Methodology}

A qualitative user evaluation study was used.
[TODO]
Method: qualitative

Instrument: Why interviews? Likert Scale and semi-structured.

Sample: Who are they? Why? Convenience Sampling. Limitation: all CS department.

An ethics submission was completed on BREO and approval granted on 14th March.
See Appendix (TODO) for the approved participant information sheet, consent form and example questions.

A total of four interviews were conducted in March 2018:
two with teaching staff and two with student representatives.
Three of the participants (A, C, E) are return participants from the requirements gathering interviews in Chapter 4.
See table \ref{table:participants-eval} for a more detailed description of these participants.

\begin{table}[!h]
	\caption{Participants in user evaluation interviews}
	\centering
	\label{table:participants-eval}
	\begin{tabularx}{\textwidth}{>{\bfseries}lX}
		Participant & Characterisation                                                                    \\
		\toprule
		Educator A  & lecturer in higher education for over 20 years, and an experienced higher education
		administrator                                                                                     \\\midrule
		Educator F  & lecturer in higher education for over 10 years                                      \\\midrule
		Student C   & a university course representative for 3 years, which involves collecting and
		communicating student feedback and attending staff-student liaison meetings                       \\\midrule
		Student E   & a university course representative for 2 years and a peer assisted learning leader
		for 1 year                                                                                        \\\bottomrule
	\end{tabularx}
\end{table}

\subsection{Interview Results and Analysis}

To better draw conclusions on the aims and objectives of this project, a Likert-type scale is used
for the first eight structured questions. This is to encourage participants to express an opinion about the guiding statements,
not for quantitative analysis. See Table \ref{table:structuredresp_eval} for the response values for these first eight questions, 
where 1 is for Strongly disagree, 2 for Disagree, 3 for Neither agree nor disagree, 4 for Agree and 5 for Strongly agree.

\begin{table}[!ht]
	\caption{Responses for structured questions in user evaluation}
	\centering
	\label{table:structuredresp_eval}
    \begin{tabularx}{\textwidth}{|c|X|c|c|c|c|}
        \hline
        & & \multicolumn{4}{c|}{Participants}\\
		\hline
		   & Statement                                                                                  & A                        & F                      & C                        & E                        \\
		\hline
		Q1 & The features of the system communicate assessment expectations very well                   & \cellcolor{Dandelion}3   & \cellcolor{green}5     & \cellcolor{SpringGreen}4 & \cellcolor{SpringGreen}4 \\
		\hline
		Q2 & The features of the system improve transparency in assessment procedures                   & \cellcolor{Dandelion}3   & \cellcolor{green}5     & \cellcolor{green}5       & \cellcolor{green}5       \\
		\hline
		Q3 & The features of the system make curriculum personalisation convenient                      & \cellcolor{green}5       & \cellcolor{green}5     & \cellcolor{green}5       & \cellcolor{SpringGreen}4 \\
		\hline
		Q4 & The system provides good (administrative/ pastoral) support for curriculum personalisation & \cellcolor{green}5       & \cellcolor{green}5     & \cellcolor{SpringGreen}4       & \cellcolor{Dandelion}3   \\
		\hline
		Q5 & The system can reduce tension and disagreements between educators and students             & \cellcolor{SpringGreen}4 & \cellcolor{Dandelion}3 & \cellcolor{Dandelion}3   & \cellcolor{SpringGreen}4 \\
		\hline
		Q6 & The system makes educational history more transparent and trustworthy                      & \cellcolor{green}5       & \cellcolor{green}5     & \cellcolor{Dandelion}3   & \cellcolor{green}5       \\
		\hline
		Q7 & The access control features of the system preserve student privacy                         & \cellcolor{SpringGreen}4 & \cellcolor{green}5     & \cellcolor{green}5       & \cellcolor{green}5       \\
		\hline
		Q8 & The system increases trust in online education providers and credentials                   & \cellcolor{SpringGreen}4 & \cellcolor{green}5     & \cellcolor{SpringGreen}4 & \cellcolor{SpringGreen}4 \\
		\hline
	\end{tabularx}
\end{table}

Overall, the participants have rated the demonstrator system positively.

The rest of the raw data from interviews (transcripts) were contextually analysed and grouped into 
observational statements (OS) and functionality suggestions (FS).
For the relevant transcript snippets for each evaluation question asked, please go to Appendix D.

\subsubsection{On Assessments}

\begin{table}[!ht]
	\begin{tabularx}{\textwidth}{|c|X|c|}
		\hline
		& Statement or Suggestion & Participant\\
		\hline
		OS1 & \textbf{The system schema encourages communication of expectations, but the quality of communication depends on the course creators} & A                 \\
		\hline
		% \multicolumn{3}{|X|}{} \\
		% \hline
	\end{tabularx}
\end{table}

\begin{table}[!ht]
	\begin{tabularx}{\textwidth}{|c|X|c|}
		\hline
		OS2 & \textbf{The assessment contract is a good reminder for students before submission} & E                 \\
		\hline
		% \multicolumn{3}{|X|}{} \\
		% \hline
	\end{tabularx}
\end{table}

\begin{table}[!ht]
	\begin{tabularx}{\textwidth}{|c|X|c|}
		\hline
		OS3 & \textbf{The assessment contract could become a source of stress} & E                 \\
		\hline
		% \multicolumn{3}{|X|}{} \\
		% \hline
	\end{tabularx}
\end{table}

\begin{table}[!ht]
	\begin{tabularx}{\textwidth}{|c|X|c|}
		\hline
		OS4 & \textbf{Assessment feedback is enhanced by the transparency of marking criteria and grade split down} & A, F, E                 \\
		\hline
		% \multicolumn{3}{|X|}{} \\
		% \hline
	\end{tabularx}
\end{table}

\begin{table}[!ht]
	\begin{tabularx}{\textwidth}{|c|X|c|}
		\hline
		OS5 & \textbf{The system help resolve tension and disagreements, not necessarily reduce them} & A, F, E                 \\
		\hline
		% \multicolumn{3}{|X|}{} \\
		% \hline
	\end{tabularx}
\end{table}

\begin{table}[!ht]
	\begin{tabularx}{\textwidth}{|c|X|c|}
		\hline
		FS1 & \textbf{Links or further explanations for knowledge required} & F                 \\
		\hline
		% \multicolumn{3}{|X|}{} \\
		% \hline
	\end{tabularx}
\end{table}

\begin{table}[!ht]
	\begin{tabularx}{\textwidth}{|c|X|c|}
		\hline
		FS2 & \textbf{Assessment appeals and special circumstances requests should be logged on the blockchain} & C                 \\
		\hline
		% \multicolumn{3}{|X|}{} \\
		% \hline
	\end{tabularx}
\end{table}

\begin{table}[!ht]
	\begin{tabularx}{\textwidth}{|c|X|c|}
		\hline
		FS3 & \textbf{Teachers should be able to override calculated grades} & C                 \\
		\hline
		\multicolumn{3}{|X|}{
			A grade override functionality that require a written reason could provide 
			flexibility to teachers but still maintaining transparency to students.
		} \\
		\hline
	\end{tabularx}
\end{table}

\subsubsection{On Curriculum Personalisation}

\begin{table}[!ht]
	\begin{tabularx}{\textwidth}{|c|X|c|}
		\hline
		OS6 & \textbf{The curriculum personalisation user interface was informative and easy to use} & A, E                 \\
		\hline
		% \multicolumn{3}{|X|}{} \\
		% \hline
	\end{tabularx}
\end{table}

\begin{table}[!ht]
	\begin{tabularx}{\textwidth}{|c|X|c|}
		\hline
		OS7 & \textbf{The direct messages with tutors and advisors is a great channel for administrative and pastoral support} & A, F, C, E                 \\
		\hline
		% \multicolumn{3}{|X|}{} \\
		% \hline
	\end{tabularx}
\end{table}

\begin{table}[!ht]
	\begin{tabularx}{\textwidth}{|c|X|c|}
		\hline
		FS4 & \textbf{The learner-staff chat records could be kept on the blockchain} & A\\
		\hline
		% \multicolumn{3}{|X|}{} \\
		% \hline
	\end{tabularx}
\end{table}

\begin{table}[!ht]
	\begin{tabularx}{\textwidth}{|c|X|c|}
		\hline
		FS5 & \textbf{The approval process should not be led by the Teacher as it reduces the final say of the student.} & E\\
		\hline
		% \multicolumn{3}{|X|}{} \\
		% \hline
	\end{tabularx}
\end{table}

% \begin{table}[!ht]
% 	\begin{tabularx}{\textwidth}{|c|X|c|}
% 		\hline
% 		FS4 & \textbf{Having a flavour of the system for MOOC platforms and a flavour of the system for Higher Education institutions.} & A, F\\
% 		\hline
% 		\multicolumn{3}{|X|}{
% 			Educator A raised the issue of "who issues the award?"
% 		} \\
% 		\hline
% 	\end{tabularx}
% \end{table}
\subsubsection{Privacy, Security and Trust}

\begin{table}[!ht]
	\begin{tabularx}{\textwidth}{|c|X|c|}
		\hline
		OS8 & \textbf{Access Controls previews are useful} & A, F, C                 \\
		\hline
		\multicolumn{3}{|X|}{
			This feature was highly praised for giving control directly in the hands of students.
		} \\
		\hline
	\end{tabularx}
\end{table}

\begin{table}[!ht]
	\begin{tabularx}{\textwidth}{|c|X|c|}
		\hline
		OS9 & \textbf{Secure storage of detailed records prevent future education history disputes} & A                 \\
		\hline
		% \multicolumn{3}{|X|}{
		% } \\
		% \hline
	\end{tabularx}
\end{table}

\begin{table}[!ht]
	\begin{tabularx}{\textwidth}{|c|X|c|}
		\hline
		FS6 & \textbf{A crowdsourced reputation building model for course modules} & A  \\
		\hline
		\multicolumn{3}{|X|}{
			A module rating system for students and other academic staff could build trust
			in the quality of content and assessments.
		} \\
		\hline
	\end{tabularx}
\end{table}

\section{Conclusion}

The demonstrator system has been a moderate success. Most of the requirement-based test cases have been passed and 
user evaluation feedback has been positive overall.
All of the participants interviewed said they would enrol into a platform like this in the future, 
praising its transparency, security and user experience.

Going back to the objectives of the project of what the demonstrator should be able to do by adopting a blockchain back-end:

\begin{itemize}
	\item improve transparency in assessments 
	\item improve curriculum personalisation
	\item improve security, privacy and access control
\end{itemize}

These objectives have been achieved well.

% Teacher participants expressed confusion around the targeted market of the application, 