%!TEX root = ../thesis.tex
% ******************************* Thesis Appendix A ****************************
\chapter{Reflection} 

\section{Project Management}

This has been the biggest academic project I have ever undertaken and 
the risk of not completing enough on time was very high.

Care was taken to make sure that I managed the project in a lean and agile method (as mentioned in Chapter 3.3 and 3.4), 
which was very successful for the design and implementation stages. 
I was able to prioritise core features and finish them as soon as possible. 
This meant that at the latter half of the project, I was able to stop working on software at any point, 
and still have a viable product.

Managing writing progress has been less successful, as every section of this paper was important and indispensable. 
I had to plan and write in a structured order. This made the kanban for writing (Table \ref{table:kanban-research}) 
less effective than originally thought, as a kanban does not help plan dates and fixed deadlines, 
and prioritisations were not really meaningful in this case. 
If I were to do another research project like this, I would use a separate waterfall project management method to manage writing.

\section{Research Skills}

A lot has been learnt about qualitative research, how to conduct them and how to analyse them.
Lots of great resources were available, such as academic guidelines and papers that critique qualitative research methods.
They have certainly opened my eyes on how to best write questions and analyse responses.

My research skills have grown but still has a lot of room for improvement. 
For example, during the evaluation interviews, participants have complained that some questions were ambiguous. 
Thankfully due to the unstructured, face-to-face nature of the activity, I could explain and correct on the spot.
Analysing qualitative data with thematic analysis techniques such as coding and theming \citep{clarke2014thematic} 
was new to me as well.

I have also taken the time out to learn and use LaTeX, a preferred document markup language in academia, 
to typeset my report.

\section{Technical Skills}

When first approaching this project I had a lot of misconceptions about blockchains and Smart Contracts, 
but mostly I was under the impression that they are extremely complex and would be too technically challenging to achieve. 
This project has definitely demystified that, and reduced my fears about approaching new technologies in general.
My ability to learn new software tools and coding languages from documentations have also improved.

I have also learnt a lot about building more complex web applications with JavaScript. Prior to this project, 
my level of understanding was very entry-level and JQuery was the only library I know of. Now I have a good understanding of
Node.js, more complex frameworks and component driven development (see Chapter 6.5.1).