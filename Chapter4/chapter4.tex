%!TEX root = ../thesis.tex
%*******************************************************************************
%****************************** Third Chapter **********************************
%*******************************************************************************
\chapter{Requirements Elicitation}

Background literature review such as Chapter 2.2 and 2.3 has driven the direction of the project and 
provided many of the functional requirements. This chapter describes the further collection of primary 
data, and provides the list of formal requirements.

\section{Primary Data}
Collecting primary data from shareholders of higher education e-learning systems would be able to:
\begin{itemize}
    \item reaffirm and further develop requirements obtained from literature
    \item obtain further requirements from the real world
\end{itemize}

\subsection{Methodology}

[TODO]

Method: qualitative

Instrument: Why interviews?

Sample: 

A total of five interviews were conducted between December 2017 and January 2018: 
two with education professionals and three with student representatives. 

Who are they?

\begin{table}[!h] 
    \caption{Participants in primary data collection interviews}
    \centering
    \label{table:participants-req}
    \begin{tabularx}{\textwidth}{>{\bfseries}lX}
        Participant & Characterisation\\
        \toprule
        Educator A & lecturer in higher education for over 20 years, has held various administrative 
        chairing roles for undergraduate programmes\\\midrule
        Educator B & lecturer in higher education for over 20 years, with research interests 
        in e-learning interactions, effectiveness and acceptance\\\midrule
        Student C & a university course representative for 3 years, which involves collecting and 
        communicating student feedback and attending staff-student liaison meetings \\\midrule
        Student D & a university peer assisted learning leader for 2 years, helping out lower level 
        students with their academic work by facilitating peer discussions, and escalating common problems
        to academic staff in debrief sessions\\\midrule
        Student E & a university course representative for 2 years\\\bottomrule
    \end{tabularx}
\end{table}

\subsection{Interviews with Education Professionals}

Professor A:
Lecturer B:

Question: What are the problems with assessments in higher education today? / 

Responses

\subsection{Interviews with Student Representatives}


What questions?

Who are they?

Responses

\section{Formal Requirements}