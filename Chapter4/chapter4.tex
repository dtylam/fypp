%!TEX root = ../thesis.tex
%*******************************************************************************
%****************************** Third Chapter **********************************
%*******************************************************************************
\chapter{Requirements Elicitation}

Background literature review in Chapter 2.1 and 2.2 has driven the direction of the project and 
provided many of the functional requirements. This chapter describes the further collection of primary 
data, and provides the list of formal user requirements.

\section{Primary Data}
Collecting primary data from shareholders of higher education e-learning systems would be able to:
\begin{itemize}
    \item reaffirm and further develop requirements obtained from literature
    \item obtain further requirements from real world painpoints and goals
\end{itemize}

\subsection{Methodology}

[TODO]

Method: qualitative

Instrument: Why interviews?

Sample: Who are they? Why?

A total of five interviews were conducted between December 2017 and February 2018: 
two with education professionals and three with student representatives. See table 
\ref{table:participants-req} for a more detailed description of these participants.

\begin{table}[!h] 
    \caption{Participants in primary data collection interviews}
    \centering
    \label{table:participants-req}
    \begin{tabularx}{\textwidth}{>{\bfseries}lX}
        Participant & Characterisation\\
        \toprule
        Educator A & lecturer in higher education for over 20 years, and an experienced higher education 
        administrator\\\midrule
        Educator B & lecturer in higher education for over 20 years, with research interests 
        in e-learning interactions, effectiveness and acceptance\\\midrule
        Student C & a university course representative for 3 years, which involves collecting and 
        communicating student feedback and attending staff-student liaison meetings \\\midrule
        Student D & a university peer assisted learning leader for 2 years, helping out lower level 
        students with their academic work by facilitating peer discussions, and escalating common problems
        to academic staff in debrief sessions\\\midrule
        Student E & a university course representative for 2 years and a peer assisted learning leader 
        for 1 year\\\bottomrule
    \end{tabularx}
\end{table}

\subsection{Interviews with Education Professionals}

Below is a list of problem statements (PS) that were raised in interviews with Educator A and B:

\subsubsection{\underline{On Assessments}}

Sample Questions: What are the problems with assessments in higher education today? / \vspace{0.25cm}\\
\textbf{PS1: Systems cannot be customised to fit assessment and grading models}\\
Many of the assessment features and functions on e-learning systems are incompatible 
with institutional requirements, or even national requirements. \textit{Educator A} pointed out how there is 
no global standard in grade point averages, and gave examples of staff:
\begin{itemize}
    \setlength\itemsep{0em}
    \item taking assessments outside the system, then putting the results back in
    \item creating grade mapping between points under the North American model and 
    grades under the British model
    \item incorporating an extra viva (oral defence) assessment step outside of e-learning systems
    \item modifying a Scandinavian system that requires double marking for all assessments 
    to making double marking optional
\end{itemize}
\textbf{PS2: Multiple systems for delivery, assessments and record keeping used in concoction without integration}\\
\textit{Educator B}: "[Teachers] have to log into all of them separately... which I find frustrating. Students 
find it frustrating as well... Blackboard Learn [a course management system] is one set of software, 
their intranet is another, Wiseflow [an e-Assessment system] is another, another login system entirely 
for e-Vision [a student record system]. Usability becomes a major pain point."\vspace{0.25cm}\\
\textbf{PS3: Oral defence may be necessary to validate automated assessments}\\
\textit{Educator A}: "[Automated assessment] is problematic... to verify that is the viva (oral defence).
When I talk to my group and ask what they got from that [automated] test they said they've got 100... 
but when I ask them to explain it to me, they could not. So the final piece of validation that is needed
before they move on from the level is to be able to question them on their answers. You can trust 
what's in the system but whether you trust the value of what's in a system depends on the viva."\vspace{0.25cm}\\
\textbf{PS4: Poor communication of assessment expectations}\\
\textit{Educator B} confirmed that the problem with transparency of assessments (as described in Chapter 2.1.1) 
does exist.\\
\textit{Educator B}: "Often it is not clear to the students what they have to do... staff [should be] making sure 
that it is clear what is expected of an assessment."\vspace{0.25cm}\\
\textbf{PS5: Lack of practice and mandatory formative feedback}\\
Students transitioning from school to higher education would experience a drop in formative assessments
(that do not affect their final grade) such as homework practices.\\
\textit{Educator B}: "[Students] are not used to not practising what they have to do over and over again before
[assessments]... and that's what they had done at school... and the students that I feel most need 
formative feedback also tends not to do them... [even though] they are the ones who would most benefit."
\vspace{0.25cm}\\
\textbf{PS6: Lack of standardisation in marking criteria and lack of automation in module review}\\
\textit{Educator B}: "In [our department] we have those assessment forms that we have to fill in, which ensures 
we at least but something for each of the [marking criteria] boxes... the marking scheme is sent to the module 
reviewer, who reads them and ensures they are clear. This is all manual... We are one of the few departments 
in the university that has anything like this. In lot of departments, these are not clearly defined 
by a long stretch."

\subsubsection{\underline{On Curriculum Personalisation}}

Sample Questions: What do you think are the current road blocks to offering more multi-disciplinary, personal, 
or even multi-institutional arrangements for higher education?\vspace{0.25cm}\\
\textbf{PS7: National regulations requiring programme outcomes and specifications makes movements 
difficult}\\
The UK higher education academic infrastructure requires all degree programme to lay out 
programme outcomes and programme specifications.\\
\textit{Educator A}: "If you want to move from [one institution] to somewhere else and you’ve got 240 credits, 
the first thing I am going to ask you is, have you read my level 1 and level 2 outcomes? Are you 
in a place where you can, if you do level 3, meet the outcomes of the programme? ... 
[Our] credit model is not the same as the North American one, so actually movement 
between [institutions or programmes], despite the government always saying they want this, 
is mitigated against by the way we conceive education, which is about programmes representing... 
what you can do when you leave a programme."
% \textit{Educator A}: "You could radically liberalise it by having a set of terminal assessments that evaluate 
% whether a student satisfy the programme outcomes"\\
\vspace{0.25cm}\\
\textbf{PS8: There is a demand in industry for multi-disciplinary degree offerings but only a few 
universities are capable of offering them}\\
Multi-disciplinary offerings seemed popular to \textit{Educator B}, and there are careers that require 
multi-disciplinary backgrounds, but there is a risk of students not making informed choices.\\
It was also argued by \textit{Educator B} that having the freedom to choose is better than universities 
defining programmes that encompasses two particular fields, because these defined degrees could have 
a low intake and not be economically viable. However, universities today seem to struggle with bureaucracy 
with students who wish to choose modules outside of their programme.\\
\textit{Educator B}: "Yes I do think there is a demand [for more multi-disciplinary offerings] and I know 
a lot of models that do work like this. The Open University and Oxford Brookes University do these 
style of degrees where students have some core modules that they have to do... surrounding that 
they can choose modules that then add up to a certain number of credits... but I think students 
need to be carefully guided through their choices and have good career advice... For example, 
the pharmaceutical industry is having a hard time finding people with a background in both IT and biology."

\subsubsection{\underline{Other Notable Issues Raised}}

\textbf{- Real world learning activities not captured in course management systems}\\
\textit{Educator A}: "There will be things that were part of the education experience that are outside of the system
such as face to face contact."\vspace{0.25cm}\\
\textbf{- Lack of student demand for geographical movement}\\
\textit{Educator A}: "People like the idea of choice... but there is no evidence that there is an appetite 
from students for this. If I offer the whole cohort the option to do a year in another country,
very few of them will take it up... it has to do with the British psyche of going to one place for 
university."\vspace{0.25cm}\\
\textbf{- Lack of support on self-regulated learning skills}\\
\textit{Educator B}: "[Students] don't realise the difference between school and university... the way they will
be expected to learn... they have to do a lot of self-motivated learning and a lot of them find that a 
huge challenge. There is not enough help and guidance for students to make that transition."\vspace{0.25cm}\\
\textbf{- The need for synoptic (cross-topics) assessments}\\
% Cross-curricular assessments here means assessing knowledge in multiple subjects in one 
% project or portfolio so as to promote learning that integrates knowledge across subjects.\\
\textit{Educator B}: "Assessments are very compartmentalised into their modules... that every module has 
its own assessments. This leads to students thinking that their learning in one module is irrelevant 
to another, whereas in the real world we draw knowledge from different areas."\vspace{0.25cm}\\
\textbf{- Multiple assessment submission deadlines at the same time for students}\\

\subsection{Interviews with Student Representatives}

Below is a list of problem statements (PS) that were raised in interviews with Educator A and B:\\
\\
\underline{\textbf{On Assessments}}
\\
What questions?

Responses

\section{Functional Requirements}

Table \ref{table:formal-reqs} lists out all the formal functional requirement statements (FR) adopted by this 
project going forward. They are related to at least one problem statements (PS) from the primary data 
gathered, or related to literature reviewed in Chapter 2. They have been ranked with the MoSCoW prioritisation
framework, which specified four levels of priority: Must Have, Should Have, Could Have, and Won’t Have 
this time \citep{agile2018moscow}. The MoSCoW levels are given from mainly a system engineering perspective 
in planning the minimum viable demonstrator product for this project, and do not necessarily reflect 
the priorities of the stakeholders.

\begin{table}[!h] 
    \caption{Prioritised list of functional requirements for this project}
    \centering
    \label{table:formal-reqs}
    \begin{tabularx}{\textwidth}{>{\bfseries}l>{\hsize=1.6\hsize}X>{\hsize=.4\hsize}Xl}
        & Requirement Statements & Related To & MoSCoW\\
        \toprule
        FR1 & The system would store learner records on a blockchain 
        & Chapter 2.2.1 & Must Have\\\midrule
        FR & The system would enforce predefined rules on assessments and evaluation with 
        smart contracts & Chapter 2.2.1, PS6 & Must Have\\\midrule
        FR & The system would enforce the provision of learning outcomes, knowledge required 
        and assessment goals at the creation or update of learning modules & Chapter 2.1.1, 
        PS4, PS6 & Must Have\\\midrule
        FR & Teachers would be able to discuss and create a customised list of courses 
        for a student, customising programme outcomes and specifications 
        & Chapter 2.1.2, PS7, PS8 & Must Have\\\midrule
        FR & Students would be able to add assessment submissions on the blockchain 
        &  & Must Have\\\midrule
        FR & The system would be able to generate certificates on the blockchain when a course 
        has been completed &  & Must Have\\\midrule
        FR & The system would provide flexible configurability for grade schema & PS1
        & Should Have\\\midrule
        FR & The system would provide one login for content delivery, assessment and 
        record keeping & PS2 & Should Have\\\midrule
        FR & The system would provide flexible configurability for multiple assessments and 
        formative assessments for a course & PS5 & Could Have\\
        \bottomrule
    \end{tabularx}
\end{table}