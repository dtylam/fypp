%!TEX root = ../thesis.tex
%*******************************************************************************
%****************************** Third Chapter **********************************
%*******************************************************************************
\chapter{Requirements Elicitation}

Background literature review such as Chapter 2.2 and 2.3 has driven the direction of the project and 
provided many of the functional requirements. This chapter describes the further collection of primary 
data, and provides the list of formal user requirements.

\section{Primary Data}
Collecting primary data from shareholders of higher education e-learning systems would be able to:
\begin{itemize}
    \item reaffirm and further develop requirements obtained from literature
    \item obtain further requirements from real world painpoints and goals
\end{itemize}

\subsection{Methodology}

[TODO]

Method: qualitative

Instrument: Why interviews?

Sample: Who are they? Why?

A total of five interviews were conducted between December 2017 and January 2018: 
two with education professionals and three with student representatives. See table 
\ref{table:participants-req} for a more detailed description of these participants.

\begin{table}[!h] 
    \caption{Participants in primary data collection interviews}
    \centering
    \label{table:participants-req}
    \begin{tabularx}{\textwidth}{>{\bfseries}lX}
        Participant & Characterisation\\
        \toprule
        Educator A & lecturer in higher education for over 20 years, has held various administrative 
        chairing roles for undergraduate programmes\\\midrule
        Educator B & lecturer in higher education for over 20 years, with research interests 
        in e-learning interactions, effectiveness and acceptance\\\midrule
        Student C & a university course representative for 3 years, which involves collecting and 
        communicating student feedback and attending staff-student liaison meetings \\\midrule
        Student D & a university peer assisted learning leader for 2 years, helping out lower level 
        students with their academic work by facilitating peer discussions, and escalating common problems
        to academic staff in debrief sessions\\\midrule
        Student E & a university course representative for 2 years\\\bottomrule
    \end{tabularx}
\end{table}

\subsection{Interviews with Education Professionals}

Below is a list of problem statements (PS) that were raised in interviews with Educator A and B:\\
\\
\underline{\textbf{On Assessments}}
\\
Questions: What are the problems with assessments in higher education today? / \\
\\
\textbf{PS1: Systems creating more manual work due to difficulties in customisation}\\
Educator A: "Many of the assessment features and functions [on e-learning systems] are incompatible 
with institutional requirements, or even national requirements"\\
\\
\underline{\textbf{On Curriculum Personalisation}}
\\
Questions: What do you think are the current road blocks to offering more multi-disciplinary, personal, 
or even multi-institutional arrangements for higher education?\\
\\
\textbf{PS1: National regulations requiring programme outcomes and specifications makes movements 
difficult}\\
Educator A: "The [UK] academic infrastructure talk about programme outcomes and programme specifications,
... So if you want to move from here to somewhere else and you’ve got 240 credits, the first thing I am 
going to ask you is, have you read my level 1 and level 2 outcomes? Are you in a place where you can, 
if you do level 3, meet the outcomes of the programme? ... whilst we have a credit model, it’s not the 
same as the North American one, so actually movement between [institutions or programmes], despite the 
government always saying they want this, is mitigated against by the way we conceive education, 
which is about programmes represent programme level outcomes, what you can do when you leave a programme."

\subsection{Interviews with Student Representatives}

Below is a list of problem statements (PS) that were raised in interviews with Educator A and B:\\
\\
\underline{\textbf{On Assessments}}
\\
What questions?

Responses

\section{Formal Requirements}

Table \ref{table:formal-reqs} lists out all the formal requirement statements (RS) adopted by this 
project going forward. They are related to at least one problem statements (PS) from the primary data 
gathered, or related to literature reviewed in Chapter 2.

\begin{table}[!h] 
    \caption{Formal list of user requirements for this project}
    \centering
    \label{table:formal-reqs}
    \begin{tabularx}{\textwidth}{>{\bfseries}lXX}
        & Requirement Statement & Related To\\
        \toprule
        RS1 & Educator A & lecturer in higher education for over 20 years, has held various administrative 
        chairing roles for undergraduate programmes\\\midrule
        RS2 & Student E & a university course representative for 2 years\\\bottomrule
    \end{tabularx}
\end{table}